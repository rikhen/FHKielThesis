%%% File encoding: UTF-8
%%% äöüÄÖÜß  <-- keine deutschen Umlaute hier? UTF-faehigen Editor verwenden!

%%% Magic Comments zum Setzen der korrekten Parameter in kompatiblen IDEs
% !TeX encoding = utf8
% !TeX program = pdflatex
% !TeX spellcheck = de_DE
% !BIB program = biber
% arara: pdflatex: { synctex: on }
% arara: biber
% arara: makeglossaries
% arara: pdflatex: { synctex: on }

\documentclass[bachelor,german]{fhthesis}
% Zulässige Optionen in [..]:
%   Typ der Arbeit: diploma, master (default), bachelor, internship
%   Hauptsprache: german (default), english
%%%----------------------------------------------------------

\usepackage{scrwfile} % Wichtig, ansonsten erscheint "No room for a new \write"
\usepackage{blindtext} % Nur zum Testen
\usepackage[layout={inline,index},innerlayout=,marginclue=true,lang=german,author=,status=draft]{fixme} % Anmerkungen einfügen
\fxusetheme{color}

%%%----------------------------------------------------------
% Verzeichnisse
%%%----------------------------------------------------------

\usepackage{fhglossaries}
\makeglossaries
\loadglsentries{glossaries/symbols}
\loadglsentries{glossaries/acronyms}

%%%----------------------------------------------------------

\RequirePackage[utf8]{inputenc}		% bei der Verw. von lualatex oder xelatex entfernen!

\graphicspath{{images/}}    % Verzeichnis mit Bildern und Grafiken
\logofile{logo}				% Logo-Datei = images/logo.pdf (\logofile{}, wenn kein Logo gewünscht)
\bibliography{bibliography}  	% Biblatex-Literaturdatei (references.bib)

%%%----------------------------------------------------------
% Angaben für die Titelei (Titelseite, Erklärung etc.)
%%%----------------------------------------------------------

%%% Einträge für ALLE Arbeiten: -----------------------------
\title{Partielle Lösungen zur allgemeinen Problematik}
\author{Peter A.\ Schlaumeier}
\studentnumber{123456} % Matrikel-Nummer
\semester{Sommersemester 2018}
\department{Maschinenwesen}% Fachbereich
\programname{Bachelor of Engineering} % Studiengang
\university{Fachhochschule Kiel} % Hochschule
\companyName{} % Firma
\placeofstudy{Kiel}
\dateofsubmission{2018}{07}{11}	% {YYYY}{MM}{DD}

%%% Zusätzlich für eine Bachelorarbeit: ---------------------
\advisor{Alois B.~Treuer, Päd.\ Phil.}
\secondexaminer{} % Zweitprüfer

%%% Restriktive Lizenformel anstatt CC (nur für Typ master) -
%\strictlicense

%%%----------------------------------------------------------
\begin{document}
%%%----------------------------------------------------------

%%%----------------------------------------------------------
\pagenumbering{Roman}
%\frontmatter                    % Titelei (röm. Seitenzahlen)
\thispagestyle{empty}
%%%----------------------------------------------------------
\maketitle

\pdfbookmark{\contentsname}{tableofcontents} % Lesezeichen für Inhaltsverzeichnis
\setcounter{page}{1} % Seitenzahlen ab hier mit #1
\pagestyle{plain}
\tableofcontents

% Verzeichnisse einfügen
\listoffigures
\listoftables

\addchap{Symbolverzeichnis}
\printglossary[type=indices, style=indices]
\printglossary[type=constants, style=constants]
\printglossary[type=latin, style=symbols]
\printglossary[type=greek, style=symbols]
\printglossary[type=dimless, style=symbols]
\addchap{Abkürzungsverzeichnis}
\printglossary[type=\acronymtype, style=acronym]

%\chapter{Vorwort}

 % Optional. Ggf. weglassen
%\chapter{Kurzfassung}



%\chapter{Abstract}


\begin{english} %switch to English language rules
This should be a 1-page (maximum) summary of your work in English.
%und hier geht dann das Abstract weiter...
\end{english}



%%%----------------------------------------------------------
\pagenumbering{arabic} % Hauptteil (ab hier arab. Seitenzahlen)
\pagestyle{headings}
%\mainmatter
%%%----------------------------------------------------------

\chapter{Einleitung}
\label{cha:Einleitung}


\chapter{Die Abschlussarbeit}
\label{cha:Abschlussarbeit}


\chapter{Zum Arbeiten mit \latex}
\label{cha:ArbeitenMitLatex}


\chapter{Abbildungen, Tabellen, Quellcode}
\label{cha:Abbildungen}



\chapter[Mathem.\ Formeln etc.]{Mathematische Formeln, Gleichungen und Algorithmen}
\label{cha:Mathematik}


\chapter[Umgang mit Literatur]{Umgang mit Literatur und anderen Quellen}
\label{cha:Literatur}


\cite{Drake1948}	% eine Quelle als Test



\chapter{Drucken der Abschlussarbeit}
\label{cha:Drucken}


\chapter{Schlussbemerkungen}
\label{cha:Schluss}



%%%----------------------------------------------------------
\pagenumbering{roman}  % ab hier römische Seitenzahlen)
\pagestyle{plain}
%%%----------------------------------------------------------

%%%----------------------------------------------------------
\appendix                                            % Anhang
%%%----------------------------------------------------------

\chapter{Technische Informationen}
\label{app:TechnischeInfos}

	% Technische Ergänzungen
\chapter{Inhalt der CD-ROM/DVD}
\label{app:cdrom}

	% Inhalt der CD-ROM/DVD
\chapter{Fragebogen}
\label{app:Fragebogen}

	% Chronologische Liste der Änderungen
\chapter{\latex-Quellkode}
\label{app:Quellcode}

	% Quelltext dieses Dokuments

%%%----------------------------------------------------------
\MakeBibliography[nosplit]                        % Quellenverzeichnis
%%%----------------------------------------------------------

%%% Messbox zur Druckkontrolle ------------------------------
\chapter*{Messbox zur Druckkontrolle}



\begin{center}
{\Large --- Druckgröße kontrollieren! ---}

\bigskip

\calibrationbox{100}{50} % Angabe der Breite/Hoehe in mm

\bigskip

{\Large --- Diese Seite nach dem Druck entfernen! ---}

\end{center}


%%%----------------------------------------------------------

%%% Übersicht mit Todo's ------------------------------------
\newpage
\listoffixmes
%%%----------------------------------------------------------
\end{document}
%%%----------------------------------------------------------
